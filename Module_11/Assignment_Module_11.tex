\documentclass[12pt, letterpaper]{../assignment}
\usepackage{graphicx}
\usepackage{courier}
\usepackage{minted}
\usepackage{amsmath}
\usepackage{polynom}
\usepackage{commath}
\usepackage{amssymb}
\usepackage{amsfonts} 
\usepackage{color}
\usepackage{cancel}
\usepackage{enumitem}
\usepackage{graphicx}
\usepackage{multirow}
\usepackage{float}
\usepackage{bm}
\usepackage{tikz}
\usetikzlibrary{shapes,arrows}
\usepackage{booktabs}
\usetikzlibrary{patterns}

% Define Theme Colors
\definecolor{light-gray}{rgb}{0.2,0.2,0.2}
\definecolor{header-blue}{rgb}{0,0,0.7}
% \definecolor{header-blue}{rgb}{0.5137,0.8353,0.9176}
\definecolor{header-blue}{rgb}{0,0.8,0.95}
\definecolor{dark-gray}{rgb}{0.1,0.1,0.1}
\pagecolor{dark-gray}
\color{white}

\usemintedstyle{monokai}
\oddsidemargin = 0pt
\exercisesheet{Module 11}{Assignment}
\student{Austin Barrilleaux}
\university{\color{header-blue}Johns Hopkins University}
\school{\color{header-blue}Whiting School of Engineering}
\courselabel{EN 535.612}
\semester{Fall 2024}
\usepackage[backend=bibtex,style=numeric,sorting=none]{biblatex}
\bibliography{reference}

\definecolor{light-gray}{rgb}{0.2,0.2,0.2}
\setminted{bgcolor=light-gray,frame=lines,rulecolor=white}
\setlength{\parindent}{0pt}

\makeatletter
\patchcmd{\minted@colorbg}{\noindent}{\medskip\noindent}{}{}
\apptocmd{\endminted@colorbg}{\par\medskip}{}{}
\makeatother

\begin{document}

\subsection*{Problem 1: Solve Ginsberg 9.28}
\subsubsection*{The absolute velocity of a particle may be represented by the components $\bm{v_x}$,
$\bm{v_y}$, and $\bm{v_z}$ relative to the axes of a moving reference system $\bm{xyz}$.
Suppose that the angular velocity $\bm{\bar{\omega}}$ of $\bm{xyz}$ and the velocity
$\bm{\bar{v}_O}$ of the origin of $\bm{xyz}$ are known as functions of time.
Derive the Gibbs-Appell equations of motion relating the quasi-velocities $\bm{\dot{\gamma}_1 = v_x}$, $\bm{\dot{\gamma}_2 = v_y}$,
and $\bm{\dot{\gamma}_3 = v_z}$ to the resultant force acting on the particle.}

Where:

$$ \bar{\omega} = 
\left<\begin{array}{c} \omega_x \\ \omega_y \\ \omega_z \end{array}\right>  $$

Given that:

$$ \bar{v} =
\left<\begin{array}{c} v_x \\ v_y \\ v_z \end{array}\right> =
\left<\begin{array}{c} \dot{\gamma}_{1}\\ \dot{\gamma}_{2}\\ \dot{\gamma}_{3} \end{array}\right> $$ 

Solving for acceleration:

\begin{equation*}
\begin{aligned}
    \bar{a} &= \frac{\partial \bar{v}}{\partial t} + \bar{\omega} \times \bar{v} \\
            &= \frac{\partial \dot{\gamma}}{\partial t} + \bar{\omega} \times \dot{\gamma} \\
            &= \left<\begin{array}{c}
{\ddot{\gamma}}_1 -{\dot{\gamma} }_2 \,\omega_z +{\dot{\gamma} }_3 \,\omega_y \\
{\ddot{\gamma}}_2 +{\dot{\gamma} }_1 \,\omega_z -{\dot{\gamma} }_3 \,\omega_x \\
{\ddot{\gamma}}_3 -{\dot{\gamma} }_1 \,\omega_y +{\dot{\gamma} }_2 \,\omega_x 
\end{array}\right>
\end{aligned}
\end{equation*}

Given that the Gibbs-Appell function for a system of particles is:

$$ S = \sum_p \frac{1}{2} m \bar{a}_p \cdot \bar{a}_p  $$

For this single particle case:

\begin{equation*}
\begin{aligned}
S &= \frac{1}{2} m \left( \bar{a}\cdot \bar{a} \right) \\
  &= \frac{1}{2}m\,{\left[
                  {{\left({\ddot{\gamma} }_3 -{\dot{\gamma} }_1 \,\omega_y +{\dot{\gamma} }_2 \,\omega_x \right)}}^2 +
                  {{\left({\ddot{\gamma} }_2 +{\dot{\gamma} }_1 \,\omega_z -{\dot{\gamma} }_3 \,\omega_x \right)}}^2+
                  {{\left({\ddot{\gamma} }_1 -{\dot{\gamma} }_2 \,\omega_z +{\dot{\gamma} }_3 \,\omega_y \right)}}^2
                  \right]}
\end{aligned}
\end{equation*}

Where the equations of motion are calculated as:

$$ \frac{\partial S}{\partial \ddot{\gamma}_j} =
\Gamma_j = \Gamma_1 $$ %=  \sum \bar{F} \cdot \bar{v}_j(q_i,t)  $$

The virtual work associated with the forces applied to the particle is:

$$ \delta W = 
\sum \bar{F} \cdot \delta \bar{x} = 
\sum_{j1}^K
\Gamma_j\ \delta \gamma_j = %\sum \bar{F} \cdot \bar{v}_{P_j}(q_i,t)\ \delta \gamma_j  $$
\sum \bar{F} \cdot \left<\begin{array}{c}
  \delta \gamma_1 \\
  \delta \gamma_2 \\
  \delta \gamma_3
\end{array}\right> $$ 

The equation of motion is solved for as:

\begin{equation*}
\begin{aligned}
    \frac{\partial S}{\partial \ddot{\gamma}} 
      &= m\left<\begin{array}{c}
{\ddot{\gamma} }_1 -{\dot{\gamma} }_2 \,\omega_z +{\dot{\gamma} }_3 \,\omega_y \\
{\ddot{\gamma} }_2 +{\dot{\gamma} }_1 \,\omega_z -{\dot{\gamma} }_3 \,\omega_x \\
{\ddot{\gamma} }_3 -{\dot{\gamma} }_1 \,\omega_y +{\dot{\gamma} }_2 \,\omega_x 
\end{array}\right>
= \Gamma_1 = 
\left<\begin{array}{c}
  \sum F_x \\
  \sum F_y \\
  \sum F_z
  \end{array}\right>= 
  \left<\begin{array}{c}
    F_x \\
    F_y \\
    F_z
    \end{array}\right>
\end{aligned}
\end{equation*}


Or:

\begin{answer}
  \begin{equation*}
    \begin{aligned}
    m\left({\ddot{\gamma} }_1 -{\dot{\gamma} }_2 \,\omega_z +{\dot{\gamma} }_3 \,\omega_y\right) &= F_x \\
    m\left({\ddot{\gamma} }_2 +{\dot{\gamma} }_1 \,\omega_z -{\dot{\gamma} }_3 \,\omega_x\right) &= F_y \\
    m\left({\ddot{\gamma} }_3 -{\dot{\gamma} }_1 \,\omega_y +{\dot{\gamma} }_2 \,\omega_x\right) &= F_z
    \end{aligned}
    \end{equation*}
\end{answer}

Where:

\begin{equation*}
    \begin{aligned}
      \dot{\gamma} =
      \left<\begin{array}{c} \dot{\gamma}_1 \\ \dot{\gamma}_2 \\ \dot{\gamma}_3  \end{array}\right>
      = v =
      \left<\begin{array}{c} v_x \\ v_y \\ v_z  \end{array}\right>
      = v_0 + \bar{\omega} \times \bar{r}
      = \left<\begin{array}{c}
        v_{0_x} -\omega_z \,y+\omega_y \,z\\
        v_{0_y} +\omega_z \,x-\omega_x \,z\\
        v_{0_z} -\omega_y \,x+\omega_x \,y
        \end{array}\right> 
    \end{aligned}
\end{equation*}

The following MATLAB script was used to solve this problem:

% \color{white}
\hspace*{6em}\inputminted[frame=leftline,fontsize=\footnotesize,lastline=38]{matlab}
{./matlab/Problem_1.m}
% \color{black} 

\subsection*{Problem 2:}
\subsubsection*{Use the Gibbs-Appell approach to find the equations of motion for this problem.}
\subsubsection*{A torque $\bm{\Gamma}$ applied to the vertical shaft of the T-bar
causes the rotation rate $\bm{\Omega}$ about the vertical axis to increase in proportion to the angle $\bm{\theta}$ by which bar BC swings outward,
that is, $\bm{\Omega = c\theta}$.
The mass of bar BC is $\bm{m_1}$ and the moment of inertia of the T-bar about its axis of rotation is $\bm{I_2}$.
Determine the equations of motion for the system, and for the torque $\bm{\Gamma}$.}

\begin{figure}[H]
    \centering
    \includegraphics[scale=0.5,frame]{images/Problem_2.png}
\end{figure}


For the purpose of this problem,
as was recommended in the office hour,
we will replace the $y$-axis of the body frame with the $z$-axis.
This will simplify the inertial to body rotation we perform later in the solution.
\\\\
First we will determine location of point $G$ in the inertial frame.
By inspection this is:

$$ \bar{r}_G = \left<\begin{array}{c}
\cos \left(\psi \right)\,{\left(L+\frac{1}{2}L\,\sin \left(\theta \right)\right)}\\
\sin \left(\psi \right)\,{\left(L+\frac{1}{2}L\,\sin \left(\theta \right)\right)}\\
-\frac{1}{2}L\,\cos \left(\theta \right)
\end{array}\right> $$

From this we can compute the velocity at point $G$:

$$ \bar{v}_G = \left<\begin{array}{c}
  \frac{1}{2}L\,\cos\left(\psi \right)\,\cos\left(\theta \right)\,\frac{\partial }{\partial t} \theta -\sin\left(\psi \right)\,\left(L+\frac{1}{2}L\,\sin\left(\theta \right)\right)\,\frac{\partial }{\partial t} \psi \\
  \frac{1}{2}L\,\cos\left(\theta \right)\,\sin\left(\psi \right)\,\frac{\partial }{\partial t} \theta +\cos\left(\psi \right)\,\left(L+\frac{1}{2}L\,\sin\left(\theta \right)\right)\,\frac{\partial }{\partial t} \psi \\
  \frac{1}{2}L\,\sin\left(\theta \right)\,\frac{\partial }{\partial t} \theta  \end{array}\right>$$

Which we can then use to compute acceleration at point $G$:

  $$ \bar{a}_G = {\scriptsize\left<\begin{array}{c} \frac{L\,\cos\left(\psi \right)\,\cos\left(\theta \right)\,\frac{\partial ^2}{\partial t^2} \theta }{2}-\frac{L\,\cos\left(\psi \right)\,\sin\left(\theta \right)\,{\left(\frac{\partial }{\partial t} \theta \right)}^2}{2}-\frac{L\,\cos\left(\psi \right)\,\left(\sin\left(\theta \right)+2\right)\,{\left(\frac{\partial }{\partial t} \psi \right)}^2}{2}-\frac{L\,\sin\left(\psi \right)\,\left(\sin\left(\theta \right)+2\right)\,\frac{\partial ^2}{\partial t^2} \psi }{2}-L\,\cos\left(\theta \right)\,\sin\left(\psi \right)\,\frac{\partial }{\partial t} \theta \,\frac{\partial }{\partial t} \psi \\
    \frac{L\,\cos\left(\theta \right)\,\sin\left(\psi \right)\,\frac{\partial ^2}{\partial t^2} \theta }{2}-\frac{L\,\sin\left(\psi \right)\,\sin\left(\theta \right)\,{\left(\frac{\partial }{\partial t} \theta \right)}^2}{2}-\frac{L\,\sin\left(\psi \right)\,\left(\sin\left(\theta \right)+2\right)\,{\left(\frac{\partial }{\partial t} \psi \right)}^2}{2}+\frac{L\,\cos\left(\psi \right)\,\left(\sin\left(\theta \right)+2\right)\,\frac{\partial ^2}{\partial t^2} \psi }{2}+L\,\cos\left(\psi \right)\,\cos\left(\theta \right)\,\frac{\partial }{\partial t} \theta \,\frac{\partial }{\partial t} \psi \\
    \frac{L\,\sin\left(\theta \right)\,\frac{\partial ^2}{\partial t^2} \theta }{2}+\frac{L\,\cos\left(\theta \right)\,{\left(\frac{\partial }{\partial t} \theta \right)}^2}{2} \end{array}\right>}$$
  
Replacing $\dot{\theta}$ and $\dot{\psi}$ with the quasi-velocity terms gives us:

$$ \bar{a}_G = {\footnotesize\left<\begin{array}{c}
  \frac{L\,\cos\left(\psi \right)\,\cos\left(\theta \right)\,\ddot{\gamma_1}}{2}-\frac{L\,\cos\left(\psi \right)\,\sin\left(\theta \right)\,{\left(\dot{\gamma_1}\right)}^2}{2}-\frac{L\,\cos\left(\psi \right)\,{\left(\dot{\gamma_2}\right)}^2\,\left(\sin\left(\theta \right)+2\right)}{2}-\frac{L\,\sin\left(\psi \right)\,\left(\sin\left(\theta \right)+2\right)\,\ddot{\gamma_2}}{2}-L\,\cos\left(\theta \right)\,\sin\left(\psi \right)\,\dot{\gamma_2}\,\dot{\gamma_1}\\
  \frac{L\,\cos\left(\theta \right)\,\sin\left(\psi \right)\,\ddot{\gamma_1}}{2}-\frac{L\,\sin\left(\psi \right)\,\sin\left(\theta \right)\,{\left(\dot{\gamma_1}\right)}^2}{2}-\frac{L\,\sin\left(\psi \right)\,{\left(\dot{\gamma_2}\right)}^2\,\left(\sin\left(\theta \right)+2\right)}{2}+\frac{L\,\cos\left(\psi \right)\,\left(\sin\left(\theta \right)+2\right)\,\ddot{\gamma_2}}{2}+L\,\cos\left(\psi \right)\,\cos\left(\theta \right)\,\dot{\gamma_2}\,\dot{\gamma_1}\\
  \frac{L\,\sin\left(\theta \right)\,\ddot{\gamma_1}}{2}+\frac{L\,\cos\left(\theta \right)\,{\left(\dot{\gamma_1}\right)}^2}{2} \end{array}\right>}$$


To convert this acceleration into the body frame, we will define the following inertial to body rotation:

$$ R_{xyz}^{XYZ} = R_z(\psi) R_y(\theta) = 
\left[\begin{array}{ccc} \cos\left(\psi \right) & -\sin\left(\psi \right) & 0\\ \sin\left(\psi \right) & \cos\left(\psi \right) & 0\\ 0 & 0 & 1 \end{array}\right]
\left[\begin{array}{ccc} \cos\left(\theta \right) & 0 & \sin\left(\theta \right)\\ 0 & 1 & 0\\ -\sin\left(\theta \right) & 0 & \cos\left(\theta \right) \end{array}\right]$$

Which simplifies to:

$$ R_{xyz}^{XYZ} = \left[\begin{array}{ccc} \cos\left(\psi \right)\,\cos\left(\theta \right) & -\sin\left(\psi \right) & \cos\left(\psi \right)\,\sin\left(\theta \right)\\ \cos\left(\theta \right)\,\sin\left(\psi \right) & \cos\left(\psi \right) & \sin\left(\psi \right)\,\sin\left(\theta \right)\\ -\sin\left(\theta \right) & 0 & \cos\left(\theta \right) \end{array}\right] $$

The transpose of this gives us the inertial to body transform:

$$ R_{XYZ}^{xyz} = \left[\begin{array}{ccc} \cos\left(\psi \right)\,\cos\left(\theta \right) & \cos\left(\theta \right)\,\sin\left(\psi \right) & -\sin\left(\theta \right)\\ -\sin\left(\psi \right) & \cos\left(\psi \right) & 0\\ \cos\left(\psi \right)\,\sin\left(\theta \right) & \sin\left(\psi \right)\,\sin\left(\theta \right) & \cos\left(\theta \right) \end{array}\right]$$

We can compute the acceleration term as:

$$ \bar{a}_G = R_{XYZ}^{xyz} \bar{a}_G =  {\footnotesize\left<
  \begin{array}{c}
    \frac{L\,\cos\left(2\,\theta \right)\,\ddot{\gamma_1}}{2}-\frac{L\,\sin\left(2\,\theta \right)\,{\left(\dot{\gamma_1}\right)}^2}{2}-\frac{L\,\sin\left(2\,\theta \right)\,{\left(\dot{\gamma_2}\right)}^2}{4}-L\,\cos\left(\theta \right)\,{\left(\dot{\gamma_2}\right)}^2\\
    L\,\ddot{\gamma_2}+\frac{L\,\sin\left(\theta \right)\,\ddot{\gamma_2}}{2}+L\,\cos\left(\theta \right)\,\dot{\gamma_2}\,\dot{\gamma_1}\\
    \frac{L\,\left(2\,{\cos\left(\theta \right)}^2-1\right)\,{\left(\dot{\gamma_1}\right)}^2}{2}-L\,\sin\left(\theta \right)\,{\left(\dot{\gamma_2}\right)}^2-\frac{L\,{\left(\dot{\gamma_2}\right)}^2}{4}+\frac{L\,\left(2\,{\cos\left(\theta \right)}^2-1\right)\,{\left(\dot{\gamma_2}\right)}^2}{4}+L\,\cos\left(\theta \right)\,\sin\left(\theta \right)\,\ddot{\gamma_1} \end{array}\right>}$$

The Gibbs-Appell function for the system is given by:

$$ S = \frac{1}{2}m \left(\bar{a}_G \cdot \bar{a}_G \right) 
+ \frac{1}{2} \bar{\alpha} \cdot \frac{\partial \bar{H}_G}{\partial t}
+ \bar{\alpha} \cdot \left( \bar{\omega} \times \bar{H}_G \right)
+ \frac{1}{2} I_2 \ddot{\psi}^2 $$

The first term of $S$ is computed as:

\begin{equation*}
  \begin{aligned}
    \frac{1}{2}m \left(\bar{a}_G \cdot \bar{a}_G \right) =
\frac{1}{2}m\,&{\left(L\,\ddot{\gamma_2}+\frac{L\,\sin\left(\theta \right)\,\ddot{\gamma_2}}{2}+L\,\cos\left(\theta \right)\,\dot{\gamma_2}\,\dot{\gamma_1}\right)}^2 \\
&+{\frac{1}{2}m\,\left(L\,\cos\left(\theta \right)\,{\left(\dot{\gamma_2}\right)}^2+\frac{L\,\sin\left(2\,\theta \right)\,{\left(\dot{\gamma_1}\right)}^2}{2}+\frac{L\,\sin\left(2\,\theta \right)\,{\left(\dot{\gamma_2}\right)}^2}{4}-\frac{L\,\cos\left(2\,\theta \right)\,\ddot{\gamma_1}}{2}\right)}^2\\
&+\frac{1}{2}m\,\left(\frac{L\,\left(2\,{\cos\left(\theta \right)}^2-1\right)\,{\left(\dot{\gamma_1}\right)}^2}{2}-L\,\sin\left(\theta \right)\,{\left(\dot{\gamma_2}\right)}^2-\frac{L\,{\left(\dot{\gamma_2}\right)}^2}{4}\right.\\
&\ \ \ \ \ \ \ \ \ \ \ \ \ \ \ +\left.\frac{L\,\left(2\,{\cos\left(\theta \right)}^2-1\right)\,{\left(\dot{\gamma_2}\right)}^2}{4}+L\,\cos\left(\theta \right)\,\sin\left(\theta \right)\,\ddot{\gamma_1}\right)^2
\end{aligned}
\end{equation*}

The angular velocity vector of the system is in the body frame is, by inspection:

$$ \bar{\omega} = \left<\begin{array}{c} -\cos\left(\gamma _{1}\right)\,\dot{\gamma_2}\\ -\dot{\gamma_1}\\ \sin\left(\gamma _{1}\right)\,\dot{\gamma_2} \end{array}\right> $$

The angular acceleration vector is therefore:

$$ \bar{\alpha} = \frac{\partial \bar{\omega}}{\partial t} + \left( \bar{\omega} \times \bar{\omega} \right) =
\left<\begin{array}{c} \sin\left(\gamma _{1}\right)\,\dot{\gamma_2}\,\dot{\gamma_1}-\cos\left(\gamma _{1}\right)\,\ddot{\gamma_2}\\ -\ddot{\gamma_1}\\ \sin\left(\gamma _{1}\right)\,\ddot{\gamma_2}+\cos\left(\gamma _{1}\right)\,\dot{\gamma_2}\,\dot{\gamma_1} \end{array}\right> $$

The inertia tensor of the bar centered at $G$ is:

$$ I = \frac{1}{12} m L^2 \left[\begin{array}{ccc}
0 & 0 & 0\\
0 & 1\\
0 & 0 & 1
\end{array}\right] $$

From this we can compute the angular momentum as:

$$ \bar{H}_G = I \bar{\omega} = \left<\begin{array}{c} 0\\ -\frac{L^2\,m\,\dot{\gamma_1}}{12}\\ \frac{L^2\,m\,\sin\left(\gamma _{1}\right)\,\dot{\gamma_2}}{12} \end{array}\right> $$

The second term in the Gibbs-Appell function is computed as:

$$ \frac{1}{2} \bar{\alpha} \cdot \frac{\partial \bar{H}_G}{\partial t} = 
\left(\frac{\sin\left(\gamma _{1}\right)\,\ddot{\gamma_2}}{2}+\frac{\cos\left(\gamma _{1}\right)\,\dot{\gamma_2}\,\dot{\gamma_1}}{2}\right)\,\left(\frac{L^2\,m\,\sin\left(\gamma _{1}\right)\,\ddot{\gamma_2}}{12}+\frac{L^2\,m\,\cos\left(\gamma _{1}\right)\,\dot{\gamma_2}\,\dot{\gamma_1}}{12}\right)+\frac{L^2\,m\,{\left(\ddot{\gamma_1}\right)}^2}{24}$$

The third term is:

$$ \bar{\alpha} \cdot \left( \bar{\omega} \times \bar{H}_G \right) = 
\frac{L^2\,m\,\cos\left(\gamma _{1}\right)\,\left(\sin\left(\gamma _{1}\right)\,\ddot{\gamma_2}+\cos\left(\gamma _{1}\right)\,\dot{\gamma_2}\,\dot{\gamma_1}\right)\,\dot{\gamma_2}\,\dot{\gamma_1}}{12}-\frac{L^2\,m\,\cos\left(\gamma _{1}\right)\,\sin\left(\gamma _{1}\right)\,{\left(\dot{\gamma_2}\right)}^2\,\ddot{\gamma_1}}{12} $$

The third term can be rewritten as:

$$ \frac{1}{2} I_2 \ddot{\psi}^2 = 
\frac{I_{2}\,{\left(\ddot{\gamma_2}\right)}^2}{2}$$

All together, we can compute the following derivatives that form the Gibbs-Appell equations as:

\begin{equation*}
  \begin{aligned}
    \frac{\partial S}{\partial \ddot{\gamma_1}} &= \frac{L^2\,m\,\ddot{\gamma_1}}{3}-\frac{L^2\,m\,\sin\left(2\,\gamma _{1}\right)\,{\left(\dot{\gamma_2}\right)}^2}{24}-\frac{L^2\,m\,\sin\left(2\,\theta \right)\,{\left(\dot{\gamma_2}\right)}^2}{8}-\frac{L^2\,m\,\cos\left(\theta \right)\,{\left(\dot{\gamma_2}\right)}^2}{2} \\
    \frac{\partial S}{\partial \ddot{\gamma_2}} &= I_{2}\,\ddot{\gamma_2}+\frac{4\,L^2\,m\,\ddot{\gamma_2}}{3}-\frac{L^2\,m\,{\cos\left(\gamma _{1}\right)}^2\,\ddot{\gamma_2}}{12}-\frac{L^2\,m\,{\cos\left(\theta \right)}^2\,\ddot{\gamma_2}}{4}+L^2\,m\,\sin\left(\theta \right)\,\ddot{\gamma_2}+L^2\,m\,\cos\left(\theta \right)\,\dot{\gamma_2}\,\dot{\gamma_1}\\
    &\ \ \ \ \ \ \ \ +\frac{L^2\,m\,\sin\left(2\,\gamma _{1}\right)\,\dot{\gamma_2}\,\dot{\gamma_1}}{12}+\frac{L^2\,m\,\cos\left(\theta \right)\,\sin\left(\theta \right)\,\dot{\gamma_2}\,\dot{\gamma_1}}{2}
  \end{aligned}
\end{equation*}

Imposing the constraint that $\dot{\gamma_2} = c\ \theta $ and $\ddot{\gamma_2} = c\ \dot{\theta} = c \dot{\gamma_1} $:

\begin{equation*}
  \begin{aligned}
    \frac{\partial S}{\partial \ddot{\gamma_1}} &= \frac{1}{3}L^2\,m\,\ddot{\gamma_1}-\frac{1}{3}L^2\,c^2\,m\,\sin\left(\theta \right)cos\left(\theta \right)\,{\gamma _{1}}^2-\frac{1}{2}L^2\,c^2\,m\,\cos\left(\theta \right)\,{\gamma _{1}}^2 \\
    \frac{\partial S}{\partial \ddot{\gamma_2}} &= I_{2}\,c\,\dot{\gamma_1}+\frac{4}{3}\,L^2\,c\,m\,\dot{\gamma_1}-\frac{1}{3}L^2\,c\,m\,{\cos\left(\theta \right)}^2\,\dot{\gamma_1}+L^2\,c\,m\,\sin\left(\theta \right)\,\dot{\gamma_1}\\
    & \ \ \ \ \ \ \ +\frac{2}{3} L^2\,c\,m\,\sin\left(\theta \right)cos\left(\theta \right)\,\gamma _{1}\,\dot{\gamma_1}+L^2\,c\,m\,\cos\left(\theta \right)\,\gamma _{1}\,\dot{\gamma_1}
  \end{aligned}
\end{equation*}

Further simplifyng:

\begin{equation*}
  \begin{aligned}
    \frac{\partial S}{\partial \ddot{\gamma_1}} &= \frac{1}{3}L^2\,m\,\ddot{\gamma_1}-L^2\,c^2\,m\,\cos\left(\theta \right)\,{\gamma _{1}}^2\left(\frac{1}{2}+\frac{1}{3}\sin\left(\theta \right) \right) = \Gamma_1\\
    \frac{\partial S}{\partial \ddot{\gamma_2}} &= I_{2}\,c\,\dot{\gamma_1}+\,L^2\,c\,m\,\left(1+\sin\left(\theta \right)+\frac{1}{3}{\sin\left(\theta \right)}^2\,\right)\dot{\gamma_1}
    + L^2\,c\,m\,\cos\left(\theta \right)\left(1+\frac{2}{3} cos\left(\theta \right)\right)\,\gamma _{1}\,\dot{\gamma_1} = \Gamma_2
  \end{aligned}
\end{equation*}

Solving for $\Gamma_j$ based on the generalized forces:

$$ \delta W = \bar{\Gamma} \cdot \delta \bar{\gamma} = \left[\begin{array}{r}
  -m\,g\,\frac{L}{2}\sin(\theta)\ \delta \gamma_1\\
  \Gamma\ \delta \gamma_2
  \end{array}\right]  $$

Therefore, the equations of motion for the system are:

\begin{equation*}
  \begin{aligned}
      \frac{1}{3}L^2\,m\,\ddot{\gamma_1}-L^2\,c^2\,m\,\cos\left(\theta \right)\,{\gamma _{1}}^2\left(\frac{1}{2}+\frac{1}{3}\sin\left(\theta \right) \right) &= -m\,g\,\frac{L}{2}\sin(\theta)\\
      I_{2}\,c\,\dot{\gamma_1}+\,L^2\,c\,m\,\left(1+\sin\left(\theta \right)+\frac{1}{3}{\sin\left(\theta \right)}^2\,\right)\dot{\gamma_1}
      + L^2\,c\,m\,\cos\left(\theta \right)\left(1+\frac{2}{3} \cos\left(\theta \right)\right)\,\gamma _{1}\,\dot{\gamma_1} &= \Gamma
  \end{aligned}
\end{equation*}

The equations of motion after simplification are:

\begin{answer}
  \begin{equation*}
    \begin{aligned}
        \frac{1}{3}\ddot{\gamma_1}-\,c^2\,\cos\left(\theta \right)\,{\gamma _{1}}^2\left(\frac{1}{2}+\frac{1}{3}\sin\left(\theta \right) \right) &= -\frac{g}{2L}\sin(\theta)\\
        \frac{1}{\,L^2\,m\,}I_{2}\dot{\gamma_1}+\left(1+\sin\left(\theta \right)+\frac{1}{3}{\sin\left(\theta \right)}^2\,\right)\dot{\gamma_1}
        + \cos\left(\theta \right)\left(1+\frac{2}{3} \cos\left(\theta \right)\right)\,\gamma _{1}\,\dot{\gamma_1} &= \frac{1}{\,L^2\,c\,m\,}\Gamma
    \end{aligned}
  \end{equation*}
\end{answer}

The following MATLAB script was used to solve this problem:

% \color{white}
\hspace*{6em}\inputminted[frame=leftline,fontsize=\footnotesize,lastline=61]{matlab}
{./matlab/Problem_2.m}
% \color{black} 

% % \color{white}
% \hspace*{6em}\inputminted[frame=leftline,fontsize=\footnotesize,lastline=51]{matlab}
% {./matlab/Problem_2.m}
% % \color{black} 
 
% % \color{white}
% \hspace*{6em}\inputminted[frame=leftline,fontsize=\footnotesize]{matlab}
% {./matlab/Q6_8.m}
% % \color{black} 

% \begin{figure}[H]
%     \centering
%     \includegraphics[scale=0.7,frame]{images/Q5_13.png}
% \end{figure}




\end{document}

