\documentclass[12pt, letterpaper]{../assignment}
\usepackage{graphicx}
\usepackage{courier}
\usepackage{minted}
\usepackage{amsmath}
\usepackage{polynom}
\usepackage{commath}
\usepackage{amssymb}
\usepackage{amsfonts} 
\usepackage{color}
\usepackage{cancel}
\usepackage{enumitem}
\usepackage{graphicx}
\usepackage{multirow}
\usepackage{float}
\usepackage{bm}
\usepackage{tikz}
\usetikzlibrary{shapes,arrows}
\usepackage{booktabs}

% Define Theme Colors
\definecolor{light-gray}{rgb}{0.2,0.2,0.2}
\definecolor{header-blue}{rgb}{0,0,0.7}
% \definecolor{header-blue}{rgb}{0.5137,0.8353,0.9176}
\definecolor{header-blue}{rgb}{0,0.8,0.95}
\definecolor{dark-gray}{rgb}{0.1,0.1,0.1}
\pagecolor{dark-gray}
\color{white}

\usemintedstyle{monokai}
\oddsidemargin = 0pt
\exercisesheet{Module 1}{Assignment}
\student{Austin Barrilleaux}
\university{\color{header-blue}Johns Hopkins University}
\school{\color{header-blue}Whiting School of Engineering}
\courselabel{EN 535.612}
\semester{Fall 2024}
\usepackage[backend=bibtex,style=numeric,sorting=none]{biblatex}
\bibliography{reference}

\definecolor{light-gray}{rgb}{0.2,0.2,0.2}
\setminted{bgcolor=light-gray,frame=lines,rulecolor=white}
\setlength{\parindent}{0pt}

\makeatletter
\patchcmd{\minted@colorbg}{\noindent}{\medskip\noindent}{}{}
\apptocmd{\endminted@colorbg}{\par\medskip}{}{}
\makeatother


\begin{document}

\subsection*{EXERCISE 2.11}
\subsubsection*{A particle slides along the hyperbolic paraboloidal surface $\bm{z = xy/2}$ such that $\bm{x = 6\cos(ku)}$,
$\bm{y = -6\sin(ku)}$, where $\bm{x}$, $\bm{y}$, and in $\bm{z}$ are in meters and $\bm{u}$ is a parameter.
Determine the path variable unit vectors,
the radius of curvature,
and the torsion of the path at the position where $\bm{ku = 2\pi/3}$}

Solving for $r'$, $r''$ and $r'''$:

\begin{equation*}
\begin{aligned}
\bar{r}' &= \left[\begin{array}{c} -6\,k\,\sin\left(k\,u\right)\\ -6\,k\,\cos\left(k\,u\right)\\ 18\,k\,{\sin\left(k\,u\right)}^2-18\,k\,{\cos\left(k\,u\right)}^2 \end{array}\right]
&= &\left[\begin{array}{c} -5.1962\,k\\ 3\,k\\ 9\,k \end{array}\right]\\
\bar{r}'' &= \left[\begin{array}{c} -6\,k^2\,\cos\left(k\,u\right)\\ 6\,k^2\,\sin\left(k\,u\right)\\ 72\,k^2\,\cos\left(k\,u\right)\,\sin\left(k\,u\right) \end{array}\right]
&= &\left[\begin{array}{c} 3\,k^2\\ 5.1962\,k^2\\ -31.1769\,k^2 \end{array}\right] \\
\bar{r}''' &= \left[\begin{array}{c} 6\,k^3\,\sin\left(k\,u\right)\\ 6\,k^3\,\cos\left(k\,u\right)\\ 72\,k^3\,{\cos\left(k\,u\right)}^2-72\,k^3\,{\sin\left(k\,u\right)}^2 \end{array}\right]
&= &\left[\begin{array}{c} 5.1962\,k^3\\ -3\,k^3\\ -36\,k^3 \end{array}\right]\\
\end{aligned}
\end{equation*}

Solving for $s'$, $s''$ and $s'''$:

\begin{equation*}
    \begin{aligned}
    {s}' &= \sqrt{\bar{r}' \cdot \bar{r}'} &= & 10.8167\,k\\
    {s}'' &= \frac{(\bar{r}' \cdot \bar{r}'')}{s'} &= & -25.9408\,k^2 \\
    {s}''' &= \frac{\bar{r}' \cdot \bar{r}'''+\bar{r}'' \cdot \bar{r}''}{s'} -
    \frac{(\bar{r}' \cdot \bar{r}'')^2}{(\bar{r}' \cdot \bar{r}')^{3/2}} &= & -2.3041\,k^3\\
    \end{aligned}
\end{equation*}

Computing the radius of curvature, $\rho$:
\begin{answer}
\begin{equation*}
    \begin{aligned}
        \rho &= \frac{{s'}^3}{\sqrt{(\bar{r}'' \cdot \bar{r}''){s'}^2 - (\bar{r}' \cdot \bar{r}'')^2}}\\
        &= 3\frac{(9\cos(4ku) + 11)^{3/2}}{\sqrt{(47 - 27\cos(4ku))}} \\
        &= 6.3917 \ \ \textbf{m}
    \end{aligned}
\end{equation*}
\end{answer}

Compute the tangent direction:
\begin{answer}
$$ \bar{e}_t = \frac{\bar{r}'}{{s'}} = 
\left[\begin{array}{r} -0.4804\\ 0.2774\\ 0.8321 \end{array}\right]$$
\end{answer}

Compute the normal direction:
\begin{answer}
$$ \bar{e}_n = \frac{\rho}{{s'}^4}\left(\bar{r}''{s'}^2 - \bar{r}' (\bar{r}' \cdot \bar{r}'')\right)
=  \left[\begin{array}{r} -0.5169\\ 0.6769\\ -0.5241 \end{array}\right]$$
\end{answer}

Compute the binormal direction:
\begin{answer}
$$ \bar{e}_b = \bar{e}_t \times \bar{e}_n = 
\left[\begin{array}{r} -0.7086\\ -0.6818\\ -0.1818 \end{array}\right]$$
\end{answer}

Knowing that:

$$ \frac{1}{\tau} = \frac{d \bar{e}_n}{d s} \cdot \bar{e}_b $$

Since the path is parameterized in u, we can write that:

$$ \frac{d \bar{e}_n}{d u} = \frac{d \bar{e}_n}{d s}\frac{d s}{d u} = 
\frac{d \bar{e}_n}{d u} s'$$

Or:

$$ \frac{d \bar{e}_n}{d s} = \frac{d \bar{e}_n}{d u} \frac{1}{s'} $$

Computing $\frac{d \bar{e}_n}{d u}$:

\begin{equation*}
\begin{aligned}
\frac{d }{d u}\left(\frac{1}{\rho}\bar{e}_n\right)
&= -\frac{\rho'}{\rho^2} \bar{e}_n + \frac{1}{\rho}\bar{e}_n'\\
&= -3\frac{r'' s' - r's''}{{s'}^4}{s''}
+ \frac{r''' s' + r''s'' - r''s'' - r's'''}{(s')^3}
\end{aligned}
\end{equation*}

We can write this as:

\begin{equation*}
    \begin{aligned}
    -\frac{\rho'}{\rho^2} \bar{e}_n + \frac{1}{\rho}\bar{e}_n'
    &= -3\frac{r'' s' - r's''}{{s'}^4}{s''}
    + \frac{r''' s' + r''s'' - r''s'' - r's'''}{(s')^3}
    \end{aligned}
\end{equation*}

If we multiply both sides by $\bar{e}_b$:

\begin{equation*}
    \begin{aligned}
    -\frac{\rho'}{\rho^2} \bar{e}_n \cdot \bar{e}_b + \frac{1}{\rho}\bar{e}_n' \cdot \bar{e}_b
    &= \left(-3\frac{r'' s' - r's''}{{s'}^4}{s''}
    + \frac{r''' s' + r''s'' - r''s'' - r's'''}{(s')^3}\right) \cdot \bar{e}_b
    \end{aligned}
\end{equation*}

Since $\bar{e}_n \cdot \bar{e}_b = 0$:

\begin{equation*}
    \begin{aligned}
        \frac{d \bar{e}_n}{d u} \cdot \bar{e}_b
    &= \left(-3\frac{r'' s' - r's''}{{s'}^4}{s''}
    + \frac{r''' s' + r''s'' - r''s'' - r's'''}{(s')^3}\right) \cdot \bar{e}_b
    \end{aligned}
\end{equation*}

Since:

$$ \frac{d \bar{e}_n}{d s} = \frac{d \bar{e}_n}{d u} \frac{1}{s'} $$

This gives us:

\begin{equation*}
    \begin{aligned}
    \frac{d \bar{e}_n}{d s} \cdot \bar{e}_b
    &= \left(-3\frac{r'' s' - r's''}{{s'}^4}{s''}
    + \frac{r''' s' + r''s'' - r''s'' - r's'''}{(s')^3}\right)\frac{1}{s'} \cdot \bar{e}_b \equiv \frac{1}{\tau}
    \end{aligned}
\end{equation*}

Evaluating this gives us:

\begin{equation*}
    \begin{aligned}
    \frac{1}{\tau} = 0.0248
    \end{aligned}
\end{equation*}

Finally:

\begin{answer}
$$ \tau = 40.3333 \ \ \textbf{m} $$
\end{answer}

\subsection*{EXERCISE 2.29}
\subsection*{In an Eulerian description of fluid flow,
particle velocity components are described as functions of the current position of a particle.
The polar velocity components of fluid particles in a certain flow are known to be $\bm{v_R = (A/R)\cos\theta}$,
$\bm{v_\theta = (A/R)\sin\theta}$, where $\bm{R}$, $\bm{\theta}    $ are the polar coordinates of the particle.
Determine the corresponding expressions for the acceleration.}

$$ v_R      = \left(\frac{A}{R}\right)\cos(\theta) $$
$$ v_\theta = \left(\frac{A}{R}\right)\sin(\theta) $$

The acceleration can be expressed by equation 2.3.14 from the textbook where the $\bar{e}_z$ term is zero:

$$ \bar{a} = \left( \ddot{R} - R \dot{\theta}^2 \right) \bar{e}_R +
\left( R \ddot{\theta} - 2 \dot{R} \dot{\theta} \right) \bar{e}_\theta $$

Since,

\begin{equation*}
\begin{aligned}
\dot{R} &= v_R = \left(\frac{A}{R}\right)\cos(\theta)\\
R \dot{\theta} &= v_\theta \ \ \rightarrow \ \ \dot{\theta} = \frac{v_\theta}{R} = \left(\frac{A}{R^2}\right)\sin(\theta) 
\end{aligned}
\end{equation*}

We can solve for the remaining undefined terms $\ddot{R}$ and $\ddot{\theta}$

\begin{equation*}
   \begin{aligned}
       \ddot{R} &= \frac{\partial \dot{R}}{\partial R} \dot{R} + \frac{\partial \dot{R}}{\partial \theta} \dot{\theta}\\
       &= -\frac{A^2\,{\cos\left(\theta \right)}^2}{R^3}
       -\frac{A^2\,{\sin\left(\theta \right)}^2}{R^3}\\
       &= -\frac{A^2}{R^3}\\
       \ddot{\theta} &= \frac{\partial \dot{\theta}}{\partial R} \dot{R} + \frac{\partial \dot{\theta}}{\partial \theta} \dot{\theta}\\
       &= -\frac{2\,A^2}{R^4} \cos\left(\theta \right)\,\sin\left(\theta \right)
       +\frac{A^2}{R^4}\cos\left(\theta \right)\,\sin\left(\theta \right)\\
       &= -\frac{A^2}{R^4} \cos\left(\theta \right)\,\sin\left(\theta \right)\\
       &= -\frac{1}{2}\frac{A^2}{R^4}\sin\left(2\,\theta \right)
   \end{aligned}
\end{equation*}

The expressions for acceleration are:

\begin{answer}
\begin{equation*}
   \begin{aligned}
       \bar{a} &= \left( \ddot{R} - R \dot{\theta}^2 \right) \bar{e}_R +
           \left( R \ddot{\theta} - 2 \dot{R} \dot{\theta} \right) \bar{e}_\theta\\
       \ddot{R} &= -\frac{A^2}{R^3}\\
       \ddot{\theta} &= -\frac{1}{2}\frac{A^2}{R^4}\sin\left(2\,\theta \right)
   \end{aligned}
\end{equation*}
\end{answer}

\subsection*{EXERCISE 2.44}

\subsection*{Observation of a small mass attached to the end of the flexible bar reveals that the path of the particle is essentially an ellipse in the horizontal plane.
The Cartesian coordinates for this motion are measured as $\bm{x = A \sin (\theta)}$, $\bm{y = 2A \cos (\theta)}$, $\bm{\theta = \omega t}$,
where $\bm{A}$ and $\bm{\omega}$ are constants.
Determine the speed, the rate of change of the speed,
and the normal acceleration at the instants when $\bm{\omega t = 0}$, $\bm{\pi/3}$, and $\bm{\pi/2}$.}

Given the definition for $x$ and $y$, we can solve for the following:

\begin{equation*}
   \begin{aligned}
       \dot{x} &= A\,\omega \,\cos\left(\omega \,t\right)\\
       \dot{y} &= -2\,A\,\omega \,\sin\left(\omega \,t\right)\\
       \ddot{x} &= -A\,\omega ^2\,\sin\left(\omega \,t\right)\\
       \ddot{y} &= -2\,A\,\omega ^2\,\cos\left(\omega \,t\right)
   \end{aligned}
\end{equation*}

Therefore the velocity and acceleration are given by,

\begin{equation*}
   \begin{aligned}
       \bar{v} &= A\,\omega \,\cos\left(\omega \,t\right)\hat{i}-2\,A\,\omega \,\sin\left(\omega \,t\right)\hat{j}\\
       &= A\,\omega\left( \,\cos\left(\omega \,t\right)\hat{i}-2 \,\sin\left(\omega \,t\right)\hat{j}\right)\\
   \end{aligned}
\end{equation*}

\begin{equation*}
   \begin{aligned}
       \bar{a} &= -A\,\omega ^2\,\sin\left(\omega \,t\right)\hat{i} -2\,A\,\omega ^2\,\cos\left(\omega \,t\right)\hat{j}\\
       &= -A\,\omega ^2\,\left(\sin\left(\omega \,t\right)\hat{i}+2\,\cos\left(\omega \,t\right)\hat{j}\right)
   \end{aligned}
\end{equation*}

Knowing that $\bar{v} = r'$ and solving for:

$$ s' = \left( \bar{r}' \cdot \bar{r}' \right)^\frac{1}{2} = A\,\omega \left( {\cos\left(\omega \,t\right)}^2+4\,{\sin\left(\omega \,t\right)}^2 \right)^\frac{1}{2}
= A\,\omega \left( 1+3\,{\sin\left(\omega \,t\right)}^2 \right)^\frac{1}{2}$$

The speed is defined as:

\begin{answer}
$$ v = A\,\omega \left( 1+3\,{\sin\left(\omega \,t\right)}^2 \right)^\frac{1}{2} $$
\end{answer}

We can solve for the tangent vector:

$$ \bar{e}_t  \equiv \frac{\bar{r}'}{s'} = \frac{1}{{\left(1+ 3\,{\sin\left(\omega \,t\right)}^2\right)}^\frac{1}{2}}
\left[\begin{array}{c} {\cos\left(\omega \,t\right)} \hat{i}\\ {2\,\sin\left(\omega \,t\right)}\hat{j} \end{array}\right]$$

Rate of change of speed is:

\begin{answer}
$$ \dot{v} = \bar{a} \cdot \bar{e}_t =
\frac{3\,A\,\omega ^2\,\cos\left(\omega \,t\right)\,\sin\left(\omega \,t\right)}{{\left(1+ 3\,{\sin\left(\omega \,t\right)}^2\right)}^\frac{1}{2}}$$
\end{answer}

Given the following definition for acceleration

$$ \bar{a} = \dot{v} \bar{e}_t + \frac{v^2}{\rho}\bar{e}_n \ \ \rightarrow  \frac{v^2}{\rho}\bar{e}_n = \bar{a} - \dot{v} \bar{e}_t  $$

The normal component of acceleration is 

\begin{equation*}
    \begin{aligned}
\bar{a} \cdot \bar{e}_n  =  \frac{v^2}{\rho} 
&= \left(\frac{4\,\omega ^4\,A^2\,\cos\left(\omega \,t\right)^2}{{\left(3\,{\sin\left(\omega \,t\right)}^2+1\right)}^2}+\frac{\omega ^4\,A^2\,\sin\left(\omega \,t\right)^2\,{\left(4\right)}^2}{{\left(3\,{\sin\left(\omega \,t\right)}^2+1\right)}^2}\right)^\frac{1}{2}\\
&= \omega^2\,A\left(\frac{4\,\cos\left(\omega \,t\right)^2+ \sin\left(\omega \,t\right)^2\,{\left(4\right)}^2}{{\left(3\,{\sin\left(\omega \,t\right)}^2+1\right)}^2}\right)^\frac{1}{2}\\
&= 2\omega^2\,A\left(\frac{\cos\left(\omega \,t\right)^2+ \sin\left(\omega \,t\right)^2\,{\left(4\right)}}{{\left(3\,{\sin\left(\omega \,t\right)}^2+1\right)}^2}\right)^\frac{1}{2}\\
&= 2\omega^2\,A\left(\frac{1+ 3\sin\left(\omega \,t\right)^2}{{\left(3\,{\sin\left(\omega \,t\right)}^2+1\right)}^2}\right)^\frac{1}{2}\\
&= 2\omega^2\,A\left(\frac{1}{{\left(3\,{\sin\left(\omega \,t\right)}^2+1\right)}}\right)^\frac{1}{2}\\
&= 2\omega^2\,A\frac{1}{{\left(3\,{\sin\left(\omega \,t\right)}^2+1\right)}^\frac{1}{2}}\\
\end{aligned}
\end{equation*}

Therefore:

\begin{answer}
\begin{equation*}
\begin{aligned}
\bar{a} \cdot \bar{e}_n
&= \frac{2\omega^2\,A}{{\left(3\,{\sin\left(\omega \,t\right)}^2+1\right)}^\frac{1}{2}}\\
\end{aligned}
\end{equation*}
\end{answer}

When $\bm{\omega t = 0}$:

\begin{answer}
\begin{equation*}
    \begin{aligned}
    v &= A\,\omega \\
    \dot{v} &= 0 \\    
    \bar{a} \cdot \bar{e}_n &= 2\,\omega^2\,{A}
    \end{aligned}
\end{equation*}
\end{answer}

When $\bm{\omega t = \pi/2}$:

\begin{answer}
\begin{equation*}
    \begin{aligned}
    v &= \frac{\sqrt{13}\,A\,\omega }{2} \\
    \dot{v} &= \frac{3\,\sqrt{3}\,\sqrt{13}\,A\,\omega ^2}{26} \\    
    \bar{a} \cdot \bar{e}_n &= {\frac{4\,\omega ^2\,A}{\sqrt{13}}}
    \end{aligned}
\end{equation*}
\end{answer}

When $\bm{\omega t = \pi/2}$:

\begin{answer}
\begin{equation*}
    \begin{aligned}
    v &= 2\,A\,\omega \\
    \dot{v} &= 0 \\    
    \bar{a} \cdot \bar{e}_n &= {\omega ^2\,{A}}
    \end{aligned}
\end{equation*}
\end{answer}

% \color{white}
% \hspace*{6em}\inputminted[frame=leftline,fontsize=\footnotesize]{matlab}
% {./matlab/B_2_18.m}
% \color{black} 

% It has the following response, which matches my analytically derived solution:

% \begin{figure}[H]
%     \centering
%     \includegraphics{matlab/B_2_18.png}
%     \caption{Response of the system}
% \end{figure}

\end{document}

